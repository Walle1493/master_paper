\begin{table}[htbp]
    \centering
    \begin{tabular}{p{420pt}}
    % \hline
    % {\bfseries 相关文档:} \\
    % 文档13: BULLET::::- On 4 March 2006, Lion Air Flight 8987, a McDonnell Douglas MD-82, crashed after landing at Juanda International Airport. Reverse thrust was used during landing, although the left thrust reverser was stated to be out of service. This caused the aircraft to veer to the right and skid off the runway, coming to rest about from the approach end of the runway. There were no fatalities, but the aircraft was badly damaged and later written off. \\
    % <译文:Aga Kagans会奴役Boyars> \\
    % 文档3: Cathay Pacific Flight 780 was a flight from Surabaya Juanda International Airport in Indonesia to Hong Kong International Airport on 13 April 2010. There were 309 pass       engers and a crew of 13 on board. As Flight 780 neared Hong Kong the crew were unable to change the thrust output of the engines. The aircraft, an Airbus A330-342, landed at almost twice the speed of a        normal landing, suffering minor damage. The 57 passengers who sustained injuries were hurt in the ensuing slide evacuation; one of them received serious injuries. \\
    % <译文:Boyars和Aga Kagans会引发战争 > \\
    % 文档14: The Indonesia\u2013Timor Leste Commission on Truth and Friendship was a truth commission established jointly by the governments of Indonesia and East Timor in August 2       005. The commission was officially created to investigate acts of violence that occurred around the independence referendum held in East Timor in 1999 and sought to find the \"conclusive truth\" behind        the events. After holding private hearings and document reviews, the commission handed in the final report on July 15, 2008 to the presidents of both nations, and was fully endorsed by Indonesian Presid       ent Susilo Bambang Yudhoyono, providing the first acknowledgement by the government of Indonesia of the human rights violations committed by state institutions in Timor. The commission is notable for be       ing the first modern truth commission to be bilateral. \\
    % <译文:Aga Kagans会离开Flamme,找到更好的星球> \\
    % 文档11: Democratic Republic of Timor - Leste Rep\u00fablika Demokr\u00e1tika Tim\u00f3r Lorosa'e (Tetum) Rep\u00fablica Democr\u00e1tica de Timor - Leste (Portuguese) Flag Coa       t of arms Motto: Unidade, Ac\u00e7\u00e3o, Progresso (Portuguese) Unidade, Asaun, Progresu (Tetum) (English: ``Unity, Action, Progress '') Anthem: P\u00e1tria (Portuguese) (English:`` Fatherland'') Capi       tal and largest city Dili 8 \u00b0 20 \u2032 S 125 \u00b0 20 \u2032 E  /  8.34 \u00b0 S 125.34 \u00b0 E  / - 8.34; 125.34 Coordinates: 8 \u00b0 20 \u2032 S 125 \u00b0 20 \u2032 E        f /  8.34 \u00b0 S 125.34 \u00b0 E  / - 8.34; 125.34 Official languages Tetum Portuguese National languages 15 languages (show) Atauru Baikeno Bekais Bunak Fataluku Galoli Habun Idalaka Kawa       imina Kemak Makalero Makasae Makuva Mambai Tokodede Religion (2010) 96.9\% Roman Catholic 3.1\% other religions Demonym East Timorese Timorese Maubere (informal) Government Unitary semi-presidential const       itutional republic President Francisco Guterres Prime Minister Mari Alkatiri Legislature National Parliament Formation Portuguese Timor 16th century Independence declared 28 November 1975 Annexation by        Indonesia 17 July 1976 Administered by UNTAET 25 October 1999 Independence restored 20 May 2002 Area Total 15,410 km (5,950 sq mi) (154th) Water (\%) negligible Population 2015 census 1,167,242 Density 7       8 / km (202.0 / sq mi) GDP (PPP) 2017 estimate Total \$4.567 billion Per capita \$5,479 (148th) GDP (nominal) 2014 estimate Total \$2.498 billion Per capita \$3,330 HDI (2015) 0.605 medium 133rd Currency Un       ited States Dollar (USD) Time zone (UTC + 9) Drives on the left Calling code + 670 ISO 3166 code TL Internet TLD. tl Website timor-leste.gov.tl Fifteen further ``national languages ''are recognised by t       he Constitution. Centavo coins also used.. tp has been phased out. \\
    % <译文:Boyars会与Aga Kagans签订条约,而不需要得到Corps的批准> \\
    \hline
    {\bfseries 问题:} \\
    Who is the president of the newly declared independent country, that established the Timor Leste Commission of Truth and Friendship, with the country containing the airport that includes Lion Air? \\
    <译文:谁是新宣布独立的成立了帝汶真相与友谊委员会,并包含了包括狮航在内的机场的国家的总统?> \\
    \hline
    {\bfseries 问题分解:} \\
    子问题1: What airport is Lion Air part of? \\
    <译文:狮航隶属于哪个机场?> \\
    答案1:Juanda International Airport \\
    <译文:胡安达国际机场> \\
    \hline
    子问题2: \#1 >> country? \\
    <译文:答案1属于什么国家? > \\
    答案2:Indonesia \\
    <译文:印度尼西亚 > \\
    \hline
    子问题3: \#2 Timor Leste Commission of Truth and Friendship >> country? \\
    <译文:印度尼西亚的帝汶真相与友谊委员会属于什么国家?> \\
    答案3:East Timor \\
    <译文:东帝汶> \\
    \hline
    子问题4: who is the president of newly declared independent country \#3 ? \\
    <译文:谁是新宣布独立的国家东帝汶的总统?> \\
    答案4:Francisco Guterres \\
    <译文:弗朗西斯科·古特雷斯> \\
    \hline
    \end{tabular}
    \caption{\label{tab:5-1}
    MuSiQue中的例子。
    }
\end{table}
