% !Mode:: "TeX:UTF-8"

% 中英文摘要
\begin{cabstract}
    % 长机器阅读理解的定义
    本文旨在探讨面向长文本的机器阅读理解技术,即让计算机程序能够自动阅读长篇文本信息,并以与人类类似的方式理解长文本,随后回答相关问题的能力。
    % MRC的重要性
    这项技术是自然语言处理领域中一个重要的研究方向,并且已经在许多应用场景中发挥了重要作用,例如智能问答、文本摘要、机器翻译和知识图谱构建等。

    % 长文本机器阅读理解的应用
    随着互联网技术的迅速发展,人们面临的文本信息越来越多,包括新闻报道、学术论文等。
    为了更快速、更准确地获取需要的信息,长文本机器阅读理解技术应运而生,并且逐渐成为研究的热点。
    % LT-MRC的重要性
    长文本阅读理解技术涉及多个子任务,如文本分段、关键信息提取、逻辑推理等。为了完成这些任务,需要采用多种自然语言处理技术,如滑动窗口、序列标注等。
    % LT-MRC的挑战
    然而,长文本阅读理解也面临一些挑战。
    首先,长文本中存在大量跨句子和跨段落的逻辑关系,例如条件关系、递进关系等,这些关系需要进行跨句子和跨段落的推理才能正确理解文本意义。
    其次,阅读和理解长文本需要消耗大量的计算资源和时间,因此如何在保证精度的前提下提高效率也是一个难点。

    % LT-MRC的解决方案1
    首先,基于目前领域内最为重要的新闻长文本语料NewsQA,本文提出了一种基于检索器和阅读器二阶段架构的方法。
    该方法基于预训练语言模型,分别实现问题与文本片段的相关性检索,以及进行序列标注,从而抽取答案文本片段。

    % LT-MRC的解决方案3
    第二,现有的长文本阅读理解架构通常只采用检索的方式来获取关键信息。
    本文另辟蹊径,通过在多跳阅读理解数据集MuSiQUe上实验,提出了一种生成式的方法,将多跳问题转化为多个单跳子问题,并依次借助阅读理解模型,提取每个子问题的关键文本段落以及答案片段。

    % LT-MRC的解决方案2
    % 最后,针对长文本多项选择语料QuALITY中,文本与多个备选答案都存在关联的情形下,仅仅依赖问题对文本进行简单的检索,往往忽略了选项存在的重要性。
    % 因此,本文利用问题与备选答案指向的稠密检索器,检索出关键信息。
    最后,针对长文本多项选择语料QuALITY中,文本与多个备选答案都存在关联的情形下,本文利用问题与备选答案指向的稠密检索器,检索出关键信息。
    同时,本文采用对比学习,以及改良过后的样本间自注意力机制,对备选答案进行更准确的语义表示,以解决备选答案之间的密切联系与区别。

    % 总结
    本文的研究目的在于探讨长文本机器阅读理解的方法,通过提取关键信息、备选答案交互以及多跳问题分解等多个角度,本文提出了三种解决方案。通过在多个开源数据集,如NewsQA、QuALITY、MuSiQue上进行实验,本文的方法取得了显著的评测指标提升,说明在长文本阅读理解方面拥有一定的实用价值。
    % 未来,我们仍希望通过不断改进算法和提高数据质量来进一步提高模型的性能和应用范围。	
	\vskip 21bp
	{\heiti\zihao{-4} 关键词:}
	机器阅读理解,
	长文本,
	稠密检索器,
	问题分解
	\begin{flushright}
		作~~~~者:董梦星
		
		指导老师:洪宇
		
	\end{flushright}
\end{cabstract}


