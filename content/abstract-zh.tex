% !Mode:: "TeX:UTF-8"

% 中英文摘要
\begin{cabstract}
    % 机器阅读理解的定义
    机器阅读理解指让计算机程序能够自动阅读文本信息,并以与人类类似的方式理解文本,然后回答相关问题的能力。
    % MRC的重要性
    机器阅读理解技术是自然语言处理领域中的一个重要研究方向,它已经在众多应用场景中发挥了重要作用,如智能问答、文本摘要、机器翻译、知识图谱构建等。

    % 长文本机器阅读理解的应用
    近年来,随着互联网技术的蓬勃发展,人们面临着越来越多的文本信息,如新闻报道、学术论文等。
    长文本机器阅读理解技术可以帮助人类更快速、更准确地获取需要的信息,提高阅读和处理文本信息的效率。
    % LT-MRC的重要性
    长文本阅读理解涉及多个子任务,主要包括文本分段、关键信息提取、逻辑推理等,需要用到多种自然语言处理技术,如滑动窗口、序列标注等。
    % LT-MRC的挑战
    目前,长文本阅读理解面临一些挑战。
    例如,长文本中存在大量的跨句子和跨段落的逻辑关系,如条件关系、递进关系等,这些关系需要进行跨句子和跨段落的推理才能正确理解文本意义。
    此外,由于长文本的复杂性,其阅读和理解需要消耗大量的计算资源和时间,因此如何在保证精度的前提下提高效率也是一个难点。

    % LT-MRC的解决方案1
    首先,基于目前领域内最为重要的新闻长文本语料NewsQA,本文提出了一种两次使用机器阅读理解模型的二阶段方法。
    具体而言,基于预训练语言模型,分别实现问题与文本片段的相关性检索,以及进行序列标注,从而抽取答案文本片段。

    % LT-MRC的解决方案2
    其次,针对长文本多项选择语料QuALITY中,文本与多个备选答案都存在关联的情形下,仅仅依赖问题对文本进行简单的检索,往往忽略了选项存在的重要性。
    因此,本文利用问题与备选答案指向的稠密检索器,检索出关键信息。
    同时,针对备选答案之间的密切联系与区别,本文采用对比学习,以及改良过后的样本间自注意力机制,对备选答案进行更准确的语义表示。

    % LT-MRC的解决方案3
    最后,现有的长文本阅读理解架构通常只采用检索的方式来获取关键信息。
    本文另辟蹊径,在多跳阅读理解数据集MuSiQUe上提出了一种生成式的方法,将复杂问题转化为多个简单子问题,并依次借助阅读理解模型,提取每个每个子问题的关键文本段落以及答案片段。

    % 总结
    本文致力于面向长文本的机器阅读理解研究,从提取关键信息、备选答案交互、复杂问题分解等多个角度出发,提出以上三种解决方案,一定程度上解决了该研究目前存在的一部分问题。
    针对开源数据集NewsQA、QuALITY、MuSiQue的实验结果表明,本文提出的方法均取得了评测指标上的提升,在长文本阅读理解方面的能力具有一定的实用价值。
    未来,我们仍希望通过不断改进算法和提高数据质量来进一步提高模型的性能和应用范围。	
	\vskip 21bp
	{\heiti\zihao{-4} 关键词:}
	阅读理解,
	长文本,
	稠密检索器,
	问题分解
	\begin{flushright}
		作~~~~者:董梦星
		
		指导老师:洪宇
		
	\end{flushright}
\end{cabstract}


