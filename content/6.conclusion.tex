\chapter{总结与展望}

\section{工作总结}
% 什么是长文本机器阅读理解?
长文本机器阅读理解是指计算机能够理解长篇文本内容并回答相关问题的能力。
利用机器阅读理解技术,搜索引擎能够直接返回用户提出的问题的正确答案,而不再受限于从检索召回的文档中推理出最终答案。
这种技术大大提高了信息检索的有效性和用户体验。
% 长文本机器阅读理解的挑战
然而,传统的基于预训练语言模型的方法面临着难以处理超过一定长度文本的挑战。
这会导致信息丢失和推理困难,从而引发一系列问题,例如如何将备选答案用于检索长文本,以及如何处理多跳问题等。
% 解决方法
为了解决这些问题,本文从三个角度提出了相应的解决方法:一是基于检索器-阅读器的二阶段架构,二是基于对比学习的选项交互,三是基于多文档问答中问题分解的研究方法。
这些方法具有针对性,能够有效地提高长文本机器阅读理解的能力。

(1)基于检索器-阅读器二阶段架构的长文本阅读理解研究

% 方法
针对滑动窗口机制在长文本阅读理解任务中存在的局限性,即由于固定长度限制而导致的长距离依赖缺失和信息丢失问题,本文提出了一种基于二阶段架构的方法。
该方法由检索器和阅读器两个模块组成,其中检索器负责对文本片段进行可回答性评分和排序,从而筛选出高置信度的证据片段,而阅读器则负责从证据片段堆中抽取答案片段。
通过这种方法,不仅可以保留关键信息,而且可以缩小答案搜索空间,提高答案抽取的效率和准确性。
% 结果
实验结果表明,该二阶段方法有效地提升了抽取式长文本阅读理解模型的能力。
与基线模型相比,该方法提升了0.6\%的F1值和1.2\%的EM值。

(2)基于问题分解的多跳长文本阅读理解研究

% 方法
本文针对多文档阅读理解任务中问题复杂性高以及传统机器阅读理解模型缺乏多跳推理能力的挑战,提出了两种基于问题分解的方法,将多跳问题转化为单跳问题。
第一种方法采用序列到序列的生成式模型,生成一系列单跳问题,并结合检索模型和阅读理解模型从多个文档中检索和抽取答案。
第二种方法则在生成当前单跳问题时,利用前驱问题和前驱答案作为额外输入来引导生成过程,并通过检索模型和阅读理解模型获取单跳答案,直到生成带有结束标志的单跳问题。
% 结果
在MuSiQue数据集上,这两种方法相较于基线模型,在证据F1指标上分别提升了2.4\%和2.3\%,在答案F1指标上分别提升了1.2\%和0.8\%。

(3)基于对比学习的多项选择长文本阅读理解研究

% 方法
本文针对多项选择阅读理解任务中的干扰选项与正确选项字面相似度高以及选项编码缺乏交互的问题,提出了一种基于对比学习和自注意力交互的方法来增强选项编码表示和区分能力。
该方法首先采用对比学习方法来训练一个选项编码器,使得不同选项之间的语义差异能够在编码空间中得到体现。
其次,利用样本内自注意力交互机制来建立各选项之间的交互关系,从而增强选项编码的区分能力。
% 结果
经过实验验证,本文提出的方法相比于基线模型,在QuALITY数据集上全部数据集上提升了6.9\%的ACC值,在困难数据集上提升了8.6\%的ACC值,具有较好的效果。


\section{工作展望}
本文针对面向长文本的机器阅读理解进行研究,并有效提高了NewsQA,MuSiQue和QuALITY等数据集相应的实验性能。
同时,在实验过程中,本文总结出以下几个亟待改进的地方:

(1)长文本阅读理解中的指代消解问题

% 问题
在长文本机器阅读理解任务中,由于文本长度巨大,存在大量指代消解问题。
本文第三章提出了一种基于检索器-阅读器二阶段架构的方法,该方法利用了预训练语言模型的能力,建立了前后文中的共指关系。
% 然而
然而,对于一些难以挖掘的指代关系,本文未能深入探讨。
% 未来
因此,在未来的研究中,可以继续研究指代消解的方法,以进一步提高本文提出的模型的性能。

(2)基于大模型和提示的多跳阅读理解

% 问题
针对多跳阅读理解问题,本文提出了两种基于问题分解的方法,以提高模型搜索证据的能力。
% 然而
尽管实验过程中已经尝试了一些生成式模型,但并未对所有模型进行全面的探索。
考虑到目前ChatGPT等大型模型的发展,这些模型或许更适合于解决问题分解任务。
% 未来
在未来的研究中,可以结合递归提示等提示技术,将复杂问题分解为多个简单子问题,并逐步回答,以进一步提高模型的准确性和效率。

(3)多种对比学习方法

% 问题
本文的第五章节采用了对比学习方法,以建立多项选择长文本阅读理解中不同选项之间的联系。
同时,本文还提出了一些衍生实验,例如构建样本内负例和增加干扰因子等。
% 然而
然而,这些实验并没有涉及多种对比学习方法。
% 未来
因此,未来的工作可以探索将其他技术如知识图谱等加入到对比学习方法中,以及从对比损失权重以及温度系数的设置等角度,进行更全面的实验。

