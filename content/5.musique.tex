\chapter{基于问题分解的多跳长文本阅读理解}
% 摘要:SeqDecomposer, StepDecomposer
% 什么是多跳阅读理解
多跳机器阅读理解是指利用多个相关文档段落进行多次推理,以实现对复杂问题的理解和回答。
相对于常规的单跳机器阅读理解,多跳机器阅读理解需要综合运用文本中的信息,以及常识和推理能力。
% 多跳阅读理解一般用到什么技术
一般来说,多跳阅读理解通常使用基于图神经网络的方法,以及基于检索的方法。
其中,图卷积网络、图注意力网络和图循环网络通过邻接矩阵建立文档间多个句子或实体之间的联系。
基于检索的方法类似于第三章提到的方法,利用文本匹配模型得到与问题相关的文档,然后进行阅读理解。
% 这些技术的缺点
然而,这些模型不擅长寻找支持证据,因为它们缺乏进行真正的多跳推理的能力。
目前的多跳阅读理解模型往往是利用了快捷方式进行求解,这意味着模型无需实际执行必要的推理步骤即可回答问题。
% 本文提出了什么方法,及其大致步骤
因此,本文基于问题分解的思想,提出了两种相关方法SeqDecomposer和StepDecomposer,用来将复杂的多跳问题分解为多个简单的单跳问题。
然后利用这些单跳问题,依次检索出相关文档作为支持证据,并对这些文档进行阅读理解,以获取每个子问题以及最终的答案。
% 实验结果
本章在多跳阅读理解数据集MuSiQue上进行了实验。
实验结果表明,SeqDecomposer和StepDecomposer与现有的检索式基线模型相比,在证据F1的性能上取得了不错的提升。

\section{引言}
% 介绍多跳阅读理解
多跳阅读理解\cite{Yang2018HotpotQAAD}(Multi-hop Reading Comprehension,MH-RC)是指需要在多个相关文档段落中进行多跳推理以实现对复杂问题的理解和回答。
不同于单跳阅读理解,多跳阅读理解更接近于人类的语言推理能力,具有广阔的应用前景但也极具挑战性。
如图~\ref{tab:5-1}~所示,给出了一个多跳阅读理解的一个例子。
在给定20个文档之后,对于提出的多跳问题“Who is the president of ... ?”,序列到序列的生成模型需要将它分解为多个单跳问题。
对于每个单跳问题,可以依次从文档中找出答案。

\begin{table}[htbp]
    \centering
    \caption{MuSiQue中的例子。}
    \begin{tabular}{p{420pt}}
    % \hline
    % {\bfseries 相关文档:} \\
    % 文档13: BULLET::::- On 4 March 2006, Lion Air Flight 8987, a McDonnell Douglas MD-82, crashed after landing at Juanda International Airport. Reverse thrust was used during landing, although the left thrust reverser was stated to be out of service. This caused the aircraft to veer to the right and skid off the runway, coming to rest about from the approach end of the runway. There were no fatalities, but the aircraft was badly damaged and later written off. \\
    % <译文:Aga Kagans会奴役Boyars> \\
    % 文档3: Cathay Pacific Flight 780 was a flight from Surabaya Juanda International Airport in Indonesia to Hong Kong International Airport on 13 April 2010. There were 309 pass       engers and a crew of 13 on board. As Flight 780 neared Hong Kong the crew were unable to change the thrust output of the engines. The aircraft, an Airbus A330-342, landed at almost twice the speed of a        normal landing, suffering minor damage. The 57 passengers who sustained injuries were hurt in the ensuing slide evacuation; one of them received serious injuries. \\
    % <译文:Boyars和Aga Kagans会引发战争 > \\
    % 文档14: The Indonesia\u2013Timor Leste Commission on Truth and Friendship was a truth commission established jointly by the governments of Indonesia and East Timor in August 2       005. The commission was officially created to investigate acts of violence that occurred around the independence referendum held in East Timor in 1999 and sought to find the \"conclusive truth\" behind        the events. After holding private hearings and document reviews, the commission handed in the final report on July 15, 2008 to the presidents of both nations, and was fully endorsed by Indonesian Presid       ent Susilo Bambang Yudhoyono, providing the first acknowledgement by the government of Indonesia of the human rights violations committed by state institutions in Timor. The commission is notable for be       ing the first modern truth commission to be bilateral. \\
    % <译文:Aga Kagans会离开Flamme,找到更好的星球> \\
    % 文档11: Democratic Republic of Timor - Leste Rep\u00fablika Demokr\u00e1tika Tim\u00f3r Lorosa'e (Tetum) Rep\u00fablica Democr\u00e1tica de Timor - Leste (Portuguese) Flag Coa       t of arms Motto: Unidade, Ac\u00e7\u00e3o, Progresso (Portuguese) Unidade, Asaun, Progresu (Tetum) (English: ``Unity, Action, Progress '') Anthem: P\u00e1tria (Portuguese) (English:`` Fatherland'') Capi       tal and largest city Dili 8 \u00b0 20 \u2032 S 125 \u00b0 20 \u2032 E  /  8.34 \u00b0 S 125.34 \u00b0 E  / - 8.34; 125.34 Coordinates: 8 \u00b0 20 \u2032 S 125 \u00b0 20 \u2032 E        f /  8.34 \u00b0 S 125.34 \u00b0 E  / - 8.34; 125.34 Official languages Tetum Portuguese National languages 15 languages (show) Atauru Baikeno Bekais Bunak Fataluku Galoli Habun Idalaka Kawa       imina Kemak Makalero Makasae Makuva Mambai Tokodede Religion (2010) 96.9\% Roman Catholic 3.1\% other religions Demonym East Timorese Timorese Maubere (informal) Government Unitary semi-presidential const       itutional republic President Francisco Guterres Prime Minister Mari Alkatiri Legislature National Parliament Formation Portuguese Timor 16th century Independence declared 28 November 1975 Annexation by        Indonesia 17 July 1976 Administered by UNTAET 25 October 1999 Independence restored 20 May 2002 Area Total 15,410 km (5,950 sq mi) (154th) Water (\%) negligible Population 2015 census 1,167,242 Density 7       8 / km (202.0 / sq mi) GDP (PPP) 2017 estimate Total \$4.567 billion Per capita \$5,479 (148th) GDP (nominal) 2014 estimate Total \$2.498 billion Per capita \$3,330 HDI (2015) 0.605 medium 133rd Currency Un       ited States Dollar (USD) Time zone (UTC + 9) Drives on the left Calling code + 670 ISO 3166 code TL Internet TLD. tl Website timor-leste.gov.tl Fifteen further ``national languages ''are recognised by t       he Constitution. Centavo coins also used.. tp has been phased out. \\
    % <译文:Boyars会与Aga Kagans签订条约,而不需要得到Corps的批准> \\
    \hline
    多跳问题: \\
    Who is the president of the newly declared independent country, that established the Timor Leste Commission of Truth and Friendship, with the country containing the airport that includes Lion Air? \\
    <译文:谁是新宣布独立的成立了帝汶真相与友谊委员会,并包含了包括狮航在内的机场的国家的总统?> \\
    \hline
    \multicolumn{1}{c}{\bfseries 问题分解} \\
    \hline
    子问题1: What airport is Lion Air part of? \\
    <译文:狮航隶属于哪个机场?> \\
    答案1:Juanda International Airport \\
    <译文:胡安达国际机场> \\
    \hline
    子问题2: \#1 >> country? \\
    <译文:答案1属于什么国家? > \\
    答案2:Indonesia \\
    <译文:印度尼西亚 > \\
    \hline
    子问题3: \#2 Timor Leste Commission of Truth and Friendship >> country? \\
    <译文:印度尼西亚的帝汶真相与友谊委员会属于什么国家?> \\
    答案3:East Timor \\
    <译文:东帝汶> \\
    \hline
    子问题4: who is the president of newly declared independent country \#3 ? \\
    <译文:谁是新宣布独立的国家东帝汶的总统?> \\
    答案4:Francisco Guterres \\
    <译文:弗朗西斯科·古特雷斯> \\
    \hline
    \end{tabular}
    \label{tab:5-1}
\end{table}
MuSiQue

% 多跳阅读理解的一般方法及其缺点
在多跳推理中,系统必须使用从多个文档中获取的信息进行推理得出最终答案。
多跳阅读理解需要对多个支持文档进行查找和推理,除了获取最终答案之外,还需要系统筛选出可供用于回答答案的支持文档。

对于多跳阅读理解,通常有两种处理手段。
% HGN
第一种是基于图神经网络的方法。
例如,HGN\cite{Fang2019HierarchicalGN}(Hierarchical Graph Network)系统使用一个层次化的图神经网络来执行多跳推理。
它通过构建一个层次图来聚合来自多个段落的线索,该图由不同粒度级别(问题、段落、句子、实体)的节点构成,其表示形式是使用预训练的BERT模型初始化的。
% SAE
第二种是基于检索的方法。
SAE\cite{Tu2019SelectAA}(Select, Answer and Explain)系统通过使用一个可解释的模型来解决这个问题,该模型可以选择最相关的文档,然后在这些文档中执行多跳推理,以回答问题。

然而,这两种方法都存在问题。
图神经网络通过一个黑盒模型,直接找到了答案。
但它并不一定能找到支持证据,因为它无需实际执行必要的推理步骤即可回答问题。
而检索的方式,只是沿用了长文本阅读理解的做法,并没有充分利用多跳的特点。

% 本文提出的方法介绍
因此,本章遵循人类回答多跳问题的方法,提出了一类基于问题分解的方法,用以将复杂的多跳问题,分解为多个简单的单跳问题。
对于这些单跳问题,依次从长文本中检索出相关文档作为支持证据,并且对这些文档进行阅读理解,来获取每个子问题以及最终的答案。

% 实验性能
本章在多跳阅读理解数据集MuSiQue上对提出的两种问题分解的方法SeqDecomposer和StepDecomposer进行了实验。
试验结果表明,与现有的检索式基线模型相比,本章提出的方法在证据F1的评价指标上取得了可观的提升,同时在答案F1上也有小幅度提升。

% 主要贡献
本章的主要贡献如下:

1.在多跳阅读理解领域,本章遵循问题分解的思路,提出两种相关的方法,SeqDecomposer和StepDecomposer,这些方法将复杂的多跳问题分解为简单的多个单跳问题,以更好的适应阅读理解任务。

2.本章在抽取式阅读理解上采用了生成式模型,对多跳问题的分解效果进行了评价。

3.针对多跳阅读理解数据集MuSiQue,本章提出的两种方法相对于检索式的基线模型,有了较为明显的提升。
(修改文字)

\section{基于问题分解的多跳长文本阅读理解}

这是面向新闻长文本的机器阅读理解。

\subsection{总体架构}

这是总体架构。

\subsection{基于端到端生成的问题分解模块}

这是基于端到端生成的问题分解模块。

\subsection{基于预训练语言模型的阅读理解模型}

这是基于预训练语言模型的阅读理解模型。

\subsection{问题分解与答案抽取的循环统一}

这是问题分解与答案抽取的循环统一。


\section{实验及结果分析}

\subsection{实验设置}

数据集、评价方法、超惨设置。

\subsection{实验结构和分析}

这是实验结果和分析。


\section{本章小结}

这是本章小结。


