\chapter{基于问题分解的多跳长文本阅读理解}
% 摘要:SeqDecomposer, StepDecomposer
% 什么是多跳阅读理解
多跳机器阅读理解是指利用多个相关文档段落进行多次推理,以实现对复杂问题的理解和回答。
相对于常规的单跳机器阅读理解,多跳机器阅读理解需要综合运用文本中的信息,以及常识和推理能力。
% 多跳阅读理解一般用到什么技术
一般来说,多跳阅读理解通常使用基于图神经网络的方法,以及基于检索的方法。
其中,图卷积网络、图注意力网络和图循环网络通过邻接矩阵建立文档间多个句子或实体之间的联系。
基于检索的方法类似于第三章提到的方法,利用文本匹配模型得到与问题相关的文档,然后进行阅读理解。
% 这些技术的缺点
然而,这些模型不擅长寻找支持证据,因为它们缺乏进行真正的多跳推理的能力。
目前的多跳阅读理解模型往往是利用了快捷方式进行求解,这意味着模型无需实际执行必要的推理步骤即可回答问题。
% 本文提出了什么方法,及其大致步骤
因此,本文基于问题分解的思想,提出了两种相关方法SeqDecomposer和StepDecomposer,用来将复杂的多跳问题分解为多个简单的单跳问题。
然后利用这些单跳问题,依次检索出相关文档作为支持证据,并对这些文档进行阅读理解,以获取每个子问题以及最终的答案。
% 实验结果
本章在多跳阅读理解数据集MuSiQue上进行了实验。
实验结果表明,SeqDecomposer和StepDecomposer与现有的检索式基线模型相比,在证据F1的性能上取得了不错的提升。

\section{引言}
% 介绍多跳阅读理解
多跳阅读理解\cite{Yang2018HotpotQAAD}(Multi-hop Reading Comprehension,MH-RC)是指需要在多个相关文档段落中进行多跳推理以实现对复杂问题的理解和回答。
不同于单跳阅读理解,多跳阅读理解更接近于人类的语言推理能力,具有广阔的应用前景但也极具挑战性。
如图~\ref{}~所示,给出了一个多跳阅读理解的一个例子。
\input{}MuSiQue

% 多跳阅读理解的一般方法及其缺点
在多跳推理中,系统必须使用从多个文档中获取的信息进行推理得出最终答案。
多跳阅读理解需要对多个支持文档进行查找和推理,除了获取最终答案之外,还需要系统筛选出可供用于回答答案的支持文档。

对于多跳阅读理解,通常有两种处理手段。
% HGN
第一种是基于图神经网络的方法。
例如,HGN\cite{Fang2019HierarchicalGN}(Hierarchical Graph Network)系统使用一个层次化的图神经网络来执行多跳推理。
它通过构建一个层次图来聚合来自多个段落的线索,该图由不同粒度级别(问题、段落、句子、实体)的节点构成,其表示形式是使用预训练的BERT模型初始化的。
% SAE
第二种是基于检索的方法。
SAE\cite{Tu2019SelectAA}(Select, Answer and Explain)系统通过使用一个可解释的模型来解决这个问题,该模型可以选择最相关的文档,然后在这些文档中执行多跳推理,以回答问题。

然而,这两种方法都存在问题。
图神经网络通过一个黑盒模型,直接找到了答案。
但它并不一定能找到支持证据,因为它无需实际执行必要的推理步骤即可回答问题。
而检索的方式,只是沿用了长文本阅读理解的做法,并没有充分利用多跳的特点。

% 本文提出的方法介绍
因此,本章遵循人类回答多跳问题的方法,提出了一类基于问题分解的方法,用以将复杂的多跳问题,分解为多个简单的单跳问题。
对于这些单跳问题,依次从长文本中检索出相关文档作为支持证据,并且对这些文档进行阅读理解,来获取每个子问题以及最终的答案。

% 实验性能
本章在多跳阅读理解数据集MuSiQue上对提出的两种问题分解的方法SeqDecomposer和StepDecomposer进行了实验。
试验结果表明,与现有的检索式基线模型相比,本章提出的方法在证据F1的评价指标上取得了可观的提升,同时在答案F1上也有小幅度提升。

% 主要贡献
本章的主要贡献如下:

1.在多跳阅读理解领域,本章遵循问题分解的思路,提出两种相关的方法,SeqDecomposer和StepDecomposer,这些方法将复杂的多跳问题分解为简单的多个单跳问题,以更好的适应阅读理解任务。

2.本章在抽取式阅读理解上采用了生成式模型,对多跳问题的分解效果进行了评价。

3.针对多跳阅读理解数据集MuSiQue,本章提出的两种方法相对于检索式的基线模型,有了较为明显的提升。
(修改文字)

\section{基于问题分解的多跳长文本阅读理解}

这是面向新闻长文本的机器阅读理解。

\subsection{总体架构}

这是总体架构。

\subsection{基于端到端生成的问题分解模块}

这是基于端到端生成的问题分解模块。

\subsection{基于预训练语言模型的阅读理解模型}

这是基于预训练语言模型的阅读理解模型。

\subsection{问题分解与答案抽取的循环统一}

这是问题分解与答案抽取的循环统一。


\section{实验及结果分析}

\subsection{实验设置}

数据集、评价方法、超惨设置。

\subsection{实验结构和分析}

这是实验结果和分析。


\section{本章小结}

这是本章小结。


